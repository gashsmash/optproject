\documentclass[10pt,a4paper]{article}		
\usepackage[T1]{fontenc} 								
%\usepackage[norsk]{babel}								
\usepackage[utf8]{inputenc}						
\usepackage{graphicx}
	\DeclareGraphicsExtensions{.eps}
\usepackage{epstopdf} 
\usepackage{epsfig}
\usepackage{pict2e}    						
\usepackage{textcomp}
\usepackage{amsmath,amssymb} 						
\usepackage{siunitx}							        
\usepackage{booktabs}
\usepackage{rotating}         
\usepackage{ifsym}
\usepackage{float}           
\usepackage[font=small,labelfont=bf]{caption}
\usepackage{pstricks}
\usepackage{hyperref}	
\setcounter{totalnumber}{5}
\renewcommand{\textfraction}{0.05}
\renewcommand{\topfraction}{0.95}
\renewcommand{\bottomfraction}{0.95}
\renewcommand{\floatpagefraction}{0.35}
\newcommand{\unit}[1]{\ensuremath{\, \mathrm{#1}}}
\newcommand{\at}[2][]{#1|_{#2}}
\DeclareMathOperator{\sech}{sech}
%%%%%%%%%%%%%%%%%%%%%%%%%%%%%%%%%%%%%%%%%%%%%%%%%%%%%%%%%%%%%%%%%%%%%%%%%
\begin{document}
\title{TMA4180 Project: Truss optimization}
\author{Student number: 732067,1-800-AMUNDSEN}
\maketitle
\begin{abstract}
\noindent
abstract
\end{abstract}
\section{Introduction}
In the simplest of terms, a truss is a collection of things bound together. The truss structure itself, is typically comprised of triangular units, with straight members, whose ends are connected at nodes. As such, they are prime candidates for mathematical modeling, and we introduce some basic physical laws governing their behavior.
\\\\
Given a straight bar $j$, between nodes $i_1$ and $i_2$ with coordinates $\boldsymbol{v}_{i1}$,$\boldsymbol{v}_{i2}\in\mathbb{R}^3$, we denote its length by $l_j = ||\boldsymbol{v}_{i2} - \boldsymbol{v}_{i1}||_{2} > 0$ and choose its spatial orientation as the directional vector $\boldsymbol{\tau}_j = (\boldsymbol{v}_{i2}-\boldsymbol{v}_{i1})/l_j$. Moreover, letting $q_j\in\mathbb{R}$ be the axial force acting on the bar $j$ such that $q_j < 0$ describes a compressing force and $q_j > 0$ corresponds to a tensile force, then we have by Hook's law:
\begin{align}
\frac{q_j}{A_j} = E_j\frac{(\boldsymbol{u}_{i2}-\boldsymbol{u}_{i1})\tau_j}{l_j} \label{eq: 1},
\end{align}
where $\boldsymbol{u}_{i2},\boldsymbol{u}_{i1}$ are the displacements of the nodes $i_2$ and $i_1$, $A_j$ the area of the bar $j$, and $E_j$ is young's modulus of the material of which the bar $j$ is made. In deriving this result, it has been assumed that the displacements are small in comparison to the length of the bar, that is: $||\boldsymbol{u}_{i2}||_2/l_j \approx 0$ and $||\boldsymbol{u}_{i1}||_2/l_j \approx 0$.
\\\\
Upon examining the individual nodes, we have by Newtons third law that:
\begin{align}
\sum_{j\in\mathcal{J}_i^{\text{out}}}q_j\boldsymbol{\tau}_j-\sum_{j\in\mathcal{J}_i^{\text{in}}}q_j\boldsymbol{\tau}_j = \boldsymbol{f}_i \label{eq: 2},
\end{align}
where $\mathcal{J}_i^{\text{out}}$ are the indices of the bars originating originating in node $i$, and $\mathcal{J}_i^{\text{in}}$ is for the bars ending in this node. $\boldsymbol{f}_i\in\mathcal{R}^3$ is the vector of external forces acting at point $i$, which is of unknown reactionary forces $\boldsymbol{f}_i^{\text{supp}}$, produced by the foundation whenever node $i$ is fixed ($\boldsymbol{u}_i = 0$), or external prescribed forces $\boldsymbol{f}_i^{\text{ext}}$ such as weight added to the truss.


\section{Optimization problems}
Given the above mathematical model of a truss, we wish to minimize the elastic strain on its structure, given vectors $\boldsymbol{q}$ and $\boldsymbol{f}^{\text{supp}}$, under the physical constraint of Newton's third law. The optimization problem may be formulated as:
\begin{align}
\min\limits_{(\boldsymbol{q},\boldsymbol{f}^{\text{supp}})}\quad&\frac{1}{2}\sum_{j=1}^m \frac{l_j q_j^2}{E_j A_j} ,\label{eq:min g}\\
\text{subject to} \quad &B\boldsymbol{q} = I_\text{supp}\boldsymbol{f}^{supp}+I_\text{ext}\boldsymbol{f}^{ext}.\label{eq:Bq}
\end{align}
\\\\
We apply the Theory of Constrained Optimization and introduce the \textit{Lagrangian function} for this system
\begin{equation*}
\mathcal{L}\left(\boldsymbol{q},\boldsymbol{f}^{\text{supp}},\boldsymbol{\lambda}\right) = \frac{1}{2}\sum_{j=1}^m \alpha_j q_{j}^2 - \sum_{i=1}^{3n}\lambda_i c_i\left(\boldsymbol{q},\boldsymbol{f}^{supp}\right),
\end{equation*}
where $\lambda_i \in \mathbb{R}$ is the \textit{Lagrange multipliers} and 
\begin{align*}
\alpha_j &= \frac{l_j}{E_j A_j}, \\
c_i\left(\boldsymbol{q},\boldsymbol{f}^\text{supp}\right) &= \left[ B\boldsymbol{q} - I_\text{supp}\boldsymbol{f}^\text{supp}-I_\text{ext}\boldsymbol{f}^\text{ext} \right]_i ,
\end{align*}
\\\\
such that each constraint, $c_i$, is the i'th row of the system (\ref{eq:Bq}). We will look for a solution, $\left(\boldsymbol{q}^*,\boldsymbol{f}^\text{supp*}\right)$, to the Optimization Problem in the \textit{feasible set} given by 

\begin{equation*}
\Omega = \lbrace \left(\boldsymbol{q},\boldsymbol{f}^\text{supp}\right) : c_i\left(\boldsymbol{q},\boldsymbol{f}^\text{supp}\right) = 0, \quad i \in \mathcal{E} \quad\text{ and }\quad c_i\left(\boldsymbol{q},\boldsymbol{f}^\text{supp}\right) \geq 0, \quad i \in \mathcal{I} \rbrace .
\end{equation*}
In our particular case, $\mathcal{I}=\varnothing$. The First-Order Necessary Conditions for the solution is given by the \textit{KKT conditions} such that the following is satisfied

\begin{align}
\nabla_{\boldsymbol{q}}\mathcal{L}\left(\boldsymbol{q}^*,\boldsymbol{f}^\text{supp*},\boldsymbol{\lambda}^*\right) &= 0 \label{eq:nabla q L} \\
\nabla_{\boldsymbol{f}^\text{supp}}\mathcal{L}\left(\boldsymbol{q}^*,\boldsymbol{f}^\text{supp*},\boldsymbol{\lambda}^*\right) &= 0 \label{eq: nabla f L} \\
c_i\left(\boldsymbol{q}^*,\boldsymbol{f}^\text{supp*}\right) &= 0, \quad \forall i \in \mathcal{E}. \label{eq: ci=0 all i in E}
\end{align}
We write out the conditions in the terms of the Optimization Problem. One can easily show that
\begin{align*}
\left(\ref{eq:nabla q L}\right) &\Rightarrow D\boldsymbol{q}^* - \sum_{i=1}^{3n}\lambda_i^*\left(B\right)_i^T = 0 \\
 &\Rightarrow D\boldsymbol{q}^* = B^T\boldsymbol{\lambda}^* \\
\left(\ref{eq: nabla f L}\right) &\Rightarrow -\sum_{i=1}^{3n}\lambda_i^*\left(I_\text{supp}\right)_i^T = 0 \\
 &\Rightarrow I_\text{supp}^T\boldsymbol{\lambda}^* = 0
\end{align*}
Thus, the First-Order Necessary Conditions of the Optimization Problem can be compactly stated as 
\begin{align*}
D\boldsymbol{q}^* &= B^T\boldsymbol{\lambda}^* \\
B\boldsymbol{q}^* &= I_\text{supp}\boldsymbol{f}^\text{supp*} + I_\text{ext}\boldsymbol{f}^\text{ext}\\
I_\text{supp}^T\boldsymbol{\lambda}^* &= 0
\end{align*} 
where $D$ is $m\times m$, $\boldsymbol{q}^*$ is $m\times 1$, $B^T$ is $m\times 3n$ and $\boldsymbol{\lambda}^*$ is $3n \times 1$. Now, with $\boldsymbol{u} = \boldsymbol{\lambda}^*$, this compact form is equivalent to the system stated in (\ref{eq: 1}) and (\ref{eq: 2}).



\begin{thebibliography}{2}
\bibitem{newtonsource}
Brynjulf Owren, \emph{TMA4212: Numerical solution of partial differential equations}, NTNU, 24 Feb, 2012.
\end{thebibliography}
\end{document}