\documentclass[10pt,a4paper]{article}		
\usepackage[T1]{fontenc} 								
%\usepackage[norsk]{babel}								
\usepackage[utf8]{inputenc}						
\usepackage{graphicx}
	\DeclareGraphicsExtensions{.eps}
\usepackage{epstopdf} 
\usepackage{epsfig}
\usepackage{pict2e}    						
\usepackage{textcomp}
\usepackage{amsmath,amssymb} 						
\usepackage{siunitx}							        
\usepackage{booktabs}
\usepackage{rotating}         
\usepackage{ifsym}
\usepackage{float}           
\usepackage[font=small,labelfont=bf]{caption}
\usepackage{pstricks}
\usepackage{hyperref}	
\setcounter{totalnumber}{5}
\renewcommand{\textfraction}{0.05}
\renewcommand{\topfraction}{0.95}
\renewcommand{\bottomfraction}{0.95}
\renewcommand{\floatpagefraction}{0.35}
\newcommand{\unit}[1]{\ensuremath{\, \mathrm{#1}}}
\newcommand{\at}[2][]{#1|_{#2}}
\DeclareMathOperator{\sech}{sech}
%%%%%%%%%%%%%%%%%%%%%%%%%%%%%%%%%%%%%%%%%%%%%%%%%%%%%%%%%%%%%%%%%%%%%%%%%
\begin{document}
\title{TMA4180 Project: Truss optimization}
\author{Student number: 732067,723984}
\maketitle
\begin{abstract}
\noindent
In this paper, we examine the necessary and sufficient conditions for solving optimization problems related to minimizing the elastic strain on truss structures. We also discuss whether or not unique solutions exist, and under what conditions we may expect a solution to exist at all.
\end{abstract}
\section{Introduction}
In the simplest of terms, a truss is a collection of things bound together. The truss structure itself, is typically comprised of triangular units, with straight members, whose ends are connected at nodes. As such, they are prime candidates for mathematical modeling, and we introduce some basic physical laws governing their behavior.
\\\\
Given a straight bar $j$, between nodes $i_1$ and $i_2$ with coordinates $\boldsymbol{v}_{i1}$,$\boldsymbol{v}_{i2}\in\mathbb{R}^3$, we denote its length by $l_j = ||\boldsymbol{v}_{i2} - \boldsymbol{v}_{i1}||_{2} > 0$ and choose its spatial orientation as the directional vector $\boldsymbol{\tau}_j = (\boldsymbol{v}_{i2}-\boldsymbol{v}_{i1})/l_j$. Moreover, letting $q_j\in\mathbb{R}$ be the axial force acting on the bar $j$ such that $q_j < 0$ describes a compressing force and $q_j > 0$ corresponds to a tensile force, then we have by Hook's law:
\begin{align}
\frac{q_j}{A_j} = E_j\frac{(\boldsymbol{u}_{i2}-\boldsymbol{u}_{i1})\boldsymbol{\tau_j}}{l_j} \label{eq: 1},
\end{align}
where $\boldsymbol{u}_{i2},\boldsymbol{u}_{i1}$ are the displacements of the nodes $i_2$ and $i_1$, $A_j>0$ the area of the bar $j$, and $E_j>0$ is young's modulus of the material of which the bar $j$ is made. In deriving this result, it has been assumed that the displacements are small in comparison to the length of the bar, that is: $||\boldsymbol{u}_{i2}||_2/l_j \approx 0$ and $||\boldsymbol{u}_{i1}||_2/l_j \approx 0$.
\\\\
Upon examining the individual nodes, we have by Newtons third law that:
\begin{align}
\sum_{j\in\mathcal{J}_i^{\text{out}}}q_j\boldsymbol{\tau}_j-\sum_{j\in\mathcal{J}_i^{\text{in}}}q_j\boldsymbol{\tau}_j = \boldsymbol{f}_i \label{eq: 2},
\end{align}
where $\mathcal{J}_i^{\text{out}}$ are the indices of the bars originating originating in node $i$, and $\mathcal{J}_i^{\text{in}}$ is for the bars ending in this node. $\boldsymbol{f}_i\in\mathcal{R}^3$ is the vector of external forces acting at point $i$, which is of unknown reactionary forces $\boldsymbol{f}_i^{\text{supp}}$, produced by the foundation whenever node $i$ is fixed ($\boldsymbol{u}_i = 0$), or external prescribed forces $\boldsymbol{f}_i^{\text{ext}}$ such as weight added to the truss.
\newpage
\noindent
If we consider an entire truss structure, with at least $n$ nodes and $m$ bars, the equations \eqref{eq: 1} and \eqref{eq: 2}, for each bar and node, will result in a system of equations:
\begin{align}
\label{datsystem}
D \boldsymbol{q} &= B^T\boldsymbol{u},\\
B \boldsymbol{q} &= I_\text{supp}\boldsymbol{f}^{\text{supp}}+I_\text{ext}\boldsymbol{f}^{\text{ext}},\\
I_\text{supp}^T\boldsymbol{u} &= 0.
\end{align}
Where we have defined the vectors: $\boldsymbol{q} = (q_1,...,q_m)^T,\ \boldsymbol{u} = (\boldsymbol{u}_1^T,...,\boldsymbol{u}_n^T)^T,\ \boldsymbol{A}=(A_1,....,A_m)^T, \ \boldsymbol{f} = (\boldsymbol{f}_1^T,...,\boldsymbol{f}_n^T),\ \boldsymbol{f}^{\text{supp}} = I_\text{supp}^T\boldsymbol{f}, \ \boldsymbol{f}^{\text{ext}} = I_\text{ext}^T\boldsymbol{f}$, and the matrices: $(D)_{i,j} = 0 \ \text{if} \ i\neq j,\ \frac{l_i q_i}{E_i A_i} \ \text{if} \ i=j$, and
\begin{align}
B_{3(i-1)+k,j} = \left\lbrace 
\begin{array}{l l l l}
-[ \boldsymbol{\tau}_j]_k & \enspace \text{if} \enspace j \in \mathcal{J}^\text{out},\\
\left\lbrace \boldsymbol{\tau}_j\right\rbrace_k & \enspace \text{if} \enspace j \in \mathcal{J}^\text{out},\\
0 & \text{otherwise},
\end{array}\right.
\end{align}
for $i=1,...,n, \ j=1,...,m, \ k=1,...,3$. Note that $I_\text{supp}$ and $I_\text{ext}$ are matrices with $\{0,1\}$ entries defined through the separation of the general force vector into external and supportive forces.
\section{Optimization problems}
\subsection{Problem 1}
Given the above mathematical model of a truss, we wish to minimize the elastic strain on its structure, with $\boldsymbol{q}$ and $\boldsymbol{f}^{\text{supp}}$, under the physical constraint of Newton's third law. The optimization problem may be formulated as:
\begin{align}
\min\limits_{(\boldsymbol{q},\boldsymbol{f}^{\text{supp}})}\quad&g(\boldsymbol{q},\boldsymbol{f}^\text{supp})=\frac{1}{2}\sum_{j=1}^m \frac{l_j q_j^2}{E_j A_j} ,\label{eq:min g}\\
\text{subject to} \quad &B\boldsymbol{q} = I_\text{supp}\boldsymbol{f}^{supp}+I_\text{ext}\boldsymbol{f}^{ext}.\label{eq:Bq}
\end{align}
We apply the Theory of Constrained Optimization and introduce the \textit{Lagrangian function} for this system:
\begin{equation*}
\mathcal{L}\left(\boldsymbol{q},\boldsymbol{f}^{\text{supp}},\boldsymbol{\lambda}\right) = \frac{1}{2}\sum_{j=1}^m \alpha_j q_{j}^2 - \sum_{i=1}^{3n}\lambda_i c_i\left(\boldsymbol{q},\boldsymbol{f}^{supp}\right),
\end{equation*}
where $\lambda_i \in \mathbb{R}$ is the \textit{Lagrange multipliers}, and 
\begin{align*}
\alpha_j &= \frac{l_j}{E_j A_j}, \\
c_i\left(\boldsymbol{q},\boldsymbol{f}^\text{supp}\right) &= \left[ B\boldsymbol{q} - I_\text{supp}\boldsymbol{f}^\text{supp}-I_\text{ext}\boldsymbol{f}^\text{ext} \right]_i ,
\end{align*}
such that each constraint, $c_i$, is the i'th row of the system (\ref{eq:Bq}). We will look for a solution, $\left(\boldsymbol{q}^*,\boldsymbol{f}^\text{supp*}\right)$, to the Optimization Problem in the \textit{feasible set} given by :
\begin{equation*}
\Omega = \lbrace \boldsymbol{q} \times \boldsymbol{f}^\text{supp} \subset \mathbb{R}^m \times \mathbb{R}^{3n} \ | \ c_i\left(\boldsymbol{q},\boldsymbol{f}^\text{supp}\right) = 0, \enspace i \in \mathcal{E},\enspace c_i\left(\boldsymbol{q},\boldsymbol{f}^\text{supp}\right) \geq 0, \enspace i \in \mathcal{I} \rbrace.
\end{equation*}
In our particular case, $\mathcal{I}=\varnothing$. Note that $\Omega$ is closed set, as it is a hyperplane in $\mathbb{R}^m \times \mathbb{R}^{3n}$.
\subsubsection{Necessary conditions}
The First-Order Necessary Conditions for the solution is given by the \textit{KKT conditions} such that the following is satisfied:
\begin{align}
\nabla_{\boldsymbol{q}}\mathcal{L}\left(\boldsymbol{q}^*,\boldsymbol{f}^\text{supp*},\boldsymbol{\lambda}^*\right) &= 0 \label{eq:nabla q L} \\
\nabla_{\boldsymbol{f}^\text{supp}}\mathcal{L}\left(\boldsymbol{q}^*,\boldsymbol{f}^\text{supp*},\boldsymbol{\lambda}^*\right) &= 0 \label{eq: nabla f L} \\
c_i\left(\boldsymbol{q}^*,\boldsymbol{f}^\text{supp*}\right) &= 0, \quad \forall i \in \mathcal{E}. \label{eq: ci=0 all i in E}
\end{align}
We write out the conditions in the terms of the Optimization Problem. \linebreak By component-wise derivation, one can easily show that
\begin{align*}
\left(\ref{eq:nabla q L}\right) &\Rightarrow D\boldsymbol{q}^* - \sum_{i=1}^{3n}\lambda_i^*\left(B\right)_i^T = 0 \\
 &\Rightarrow D\boldsymbol{q}^* = B^T\boldsymbol{\lambda}^* \\
\left(\ref{eq: nabla f L}\right) &\Rightarrow -\sum_{i=1}^{3n}\lambda_i^*\left(I_\text{supp}\right)_i^T = 0 \\
 &\Rightarrow I_\text{supp}^T\boldsymbol{\lambda}^* = 0
\end{align*}
Thus, the First-Order Necessary Conditions of the Optimization Problem can be compactly stated as 
\begin{align*}
D\boldsymbol{q}^* &= B^T\boldsymbol{\lambda}^* \\
B\boldsymbol{q}^* &= I_\text{supp}\boldsymbol{f}^\text{supp*} + I_\text{ext}\boldsymbol{f}^\text{ext}\\
I_\text{supp}^T\boldsymbol{\lambda}^* &= 0
\end{align*} 
where $D$ is $m\times m$, $\boldsymbol{q}^*$ is $m\times 1$, $B^T$ is $m\times 3n$ and $\boldsymbol{\lambda}^*$ is $3n \times 1$. Letting $\boldsymbol{u} = \boldsymbol{\lambda}^*$, this compact form is equivalent to the system stated in (\ref{eq: 1}) and (\ref{eq: 2}), so the system at least constitutes the necessary conditions. However, it remains to prove that they are sufficient.
\subsubsection{Sufficient conditions}
It can be shown that the KKT conditions for minimization are sufficient under convexity if the cost function is convex, and the inequality and equality constraints are concave and affine, respectively, assuming a KKT point exists\footnote{Theorem: KKT conditions are sufficient under convexity, TMA4180 Notes, URL: \url{http://www.math.ntnu.no/emner/TMA4180/2014v/Notes/KKT_suff.pdf}}. Obviously, the equality constraints are affine, and with no inequality constraints, we only need to prove that \eqref{eq:min g} is convex.
\\\\
A function is convex if its corresponding hessian is positive semi-definite on the interior of the convex set. The hessian of the cost function $g$ in \eqref{eq:min g} is easily shown to be:
\begin{align}
H(g)_{(m+3n_s) \times (m+3n_s)} = 
\begin{pmatrix}
D & 0 \\
0 & 0
\end{pmatrix}
\end{align}
which by the definition of D implies that the hessian is positive semi-definite.
Hence, $g$ is convex in $\Omega$. Thus the system (\ref{eq: 1}) and (\ref{eq: 2}) constitutes the necessary and sufficient conditions for optimality in \eqref{eq:min g} and \eqref{eq:Bq}.
\subsubsection{Existence and uniqueness of solutions}
Assuming that the feasible set is non-empty, we need to show that our optimization problem admits at least one solution. We use theorem 2\footnote{Theorem 2, TMA4180 Notes, URL: \url{http://www.math.ntnu.no/emner/TMA4180/2014v/Notes/note01.pdf}} which states that if $\Omega$ is a non-empty and closed set, and the cost function $g$ is coersive and lower semi-continuous, then the optimization problem admits at least one solution. It is obvious that $g(\boldsymbol{p},\boldsymbol{f}^\text{supp}) \rightarrow +\infty$ as $||\boldsymbol{q}||_2 \rightarrow +\infty$, and thus it is a coersive function. To see that it lower semi-continuous, we construct the lower level set $\Omega_\alpha$ = $\{\Omega \ | \ g(\boldsymbol{q},\boldsymbol{f}^\text{supp}) \leq \alpha \}$. We then have that:
\begin{align}
\Omega_\alpha = \left\lbrace 
\begin{array}{l l l}
\varnothing & \enspace \alpha < 0\\
\Omega & \enspace \alpha \geq 0,
\end{array}\right.
\end{align}
all of which are closed sets in $\Omega$. Thus, by definition, $g$ is l.s.c.
\\\\
To show uniqueness of this solution, we first consider the case $\boldsymbol{f}^{\text{ext}} = 0$. We then have that for some $\boldsymbol{q},\boldsymbol{f}^{\text{supp}}\in\Omega$:
\begin{align*}
\boldsymbol{q}^T D \boldsymbol{q} = \boldsymbol{u}^T I_{\text{supp}} = 0,
\end{align*}
which by the KKT conditions with $\boldsymbol{\lambda} = \boldsymbol{u}$ implies that both $\boldsymbol{q} = 0$ and $\boldsymbol{f}^{\text{supp}}=0$. Thus the system of equations:
\begin{align}
\begin{pmatrix}
D & -B^T & 0\\
B & 0 & -I_\text{supp}\\
0 & I_\text{supp}^T & 0 
\end{pmatrix}
\begin{pmatrix}
\boldsymbol{q}\\
\boldsymbol{\lambda}\\
\boldsymbol{f}_\text{supp}
\end{pmatrix}
=
\begin{pmatrix}
0\\
I_\text{ext}\boldsymbol{f}_\text{ext}\\
0
\end{pmatrix}
\end{align}
only contains the zero vectors in its null space, given that the row vectors $B$ are linearly independent (LICQ condition). This in turn implies that the solution to the optimization problem is unique.
\subsection{Additional constraints}
We introduce the additional constrains on Problem 1 that,
\begin{align}
\label{problem2}
&\sum_{j=1}^M \rho_j l_j A_j \leq M,\\
\label{problem2A}
&\underline{A}_j \leq A_j \leq \overline{A}_j \quad j=1,...,m,
\end{align}
where $M > 0$ is the maximal allowable weight on the truss structure, $\rho_j > 0$ is the density of the material in the bar $j$, and $0 < \underline{A}_j <\overline{A}_j \leq +\infty$ are lower and upper bounds on the bar cross-sectional areas. With the addition of these constraints, we now consider the problem of minimizing $g(\boldsymbol{q},\boldsymbol{f}^{supp},\boldsymbol{A})$ over the area as well.
\subsubsection{Existence and uniqueness of solutions}
The objective function is still coersive. However, the feasible set has changed, since $\mathcal{I} \neq \varnothing$, and so we need to check if $g$ is still lower semi-continuous on the new $\Omega$. Since $\Omega$ is still a closed set, because the set of $\boldsymbol{f}^{supp},\boldsymbol{q}$ and now also $\boldsymbol{A}$ which satisfy the inequality and equality constraints is still closed, we find that $g$ is lower semi-continuous as before. This is easily seen upon the construction of $\Omega_\alpha$. Then, by theorem 2, there exists at least one optimal solution, under the assumption that the feasible set is non-empty.
\\\\
Suppose that the lower bound $\underline{A}_j=0$ for some $j$, we then adopt the convention that $q_j^2/0 = +\infty$, if $q_j\neq0$, and $0$ whenever $q_j=0$. Then $g$ is still coersive, and to see if its still lower semi-continuous at this point we look at the pointwise definition for some function $f(x)$:
\begin{align}
\liminf_{x \rightarrow x_0} f(x) \geq f(x_0).
\end{align}
Clearly, for any neighborhood $U$ of $q_j = 0$, we must have that $g(q_j\neq0,...) \geq g(q_j = 0,...) - \epsilon$ for any $\epsilon > 0$. Likewise, starting in some $q_j \neq 0$ we also have that $g(q_j\neq0,....) \rightarrow +\infty$ as $q_j \rightarrow q_j^\dagger$ where $q_j^\dagger\in U$, since $g(q_j^\dagger \neq 0,...)=+\infty$ Thus, $g$ is l.s.c.
\subsubsection{Necessary Conditions}
Letting $\overline{A}_j = +\infty$ and $\underline{A}_j = 0$ for all $j$. Then, whenever $A_j = +\infty$ or $A_j = 0$, its constraint in \eqref{problem2A} is active. However, if indeed $A_j = +\infty$ then the inequality \eqref{problem2} cannot possibly be satisfied, which in turn means that the feasible set is empty whenever the upper bound constraint on A is active. Additionally, whenever $A_j = 0$, we must fix $q_j = 0$ or else the objective function is infinitely large, in which case, it makes little sense to minimize it\footnote{Infinity Doge, URL: \url{http://4.bp.blogspot.com/-2RxHjA9d7lU/Unk6iKQU6kI/AAAAAAAAZus/55bG95S2Cqs/s1600/doge+gif+infinity+dr+heckle+funny+wtf+gifs.gif}}...
\\\\
The First-Order Necessary Conditions of Problem 1 will then be modified to:
\begin{align}\label{additionalgay}
\nabla_{\boldsymbol{q}}\mathcal{L}\left(\boldsymbol{q}^*,\boldsymbol{f}^\text{supp*},\boldsymbol{A}^*,\boldsymbol{\lambda}^*\right) &= 0, \\
\nabla_{\boldsymbol{f}^\text{supp}}\mathcal{L}\left(\boldsymbol{q}^*,\boldsymbol{f}^\text{supp*},\boldsymbol{A}^*,\boldsymbol{\lambda}^*\right) &= 0,\\
\nabla_{\boldsymbol{A}}\mathcal{L}\left(\boldsymbol{q}^*,\boldsymbol{f}^\text{supp*},\boldsymbol{A}^*,\boldsymbol{\lambda}^*\right) &= 0, \\
c_i\left(\boldsymbol{q}^*,\boldsymbol{f}^\text{supp*},\boldsymbol{A}^*\right) &= 0, \enspace \forall i \in \mathcal{E},\\
\lambda_i^* &\geq 0, \enspace \forall i\in\mathcal{I},\\
\lambda_i^* c_i(\boldsymbol{q}^*,\boldsymbol{f}_\text{supp}^*,\boldsymbol{A}^*) &= 0, \enspace \forall i\in\mathcal{I}\cup\mathcal{E},\\
c_i(\boldsymbol{q}^*,\boldsymbol{f}_\text{supp}^*,\boldsymbol{A}^*) &\geq 0, \enspace \forall i\in\mathcal{I}.\label{finalgay}
\end{align}
\newpage \noindent
\subsubsection{Sufficient Conditions}
As before, we use that the KKT conditions for minimization are sufficient under convexity, and define that the linear inequality constraints as concave. Then we only need to show that the objective function is convex. 
The hessian for $g$ will be:
\begin{align}
H(g) = 
\begin{pmatrix}
D & 0 & \cdots & -\frac{l_1 q_1}{E_1 A_1^2} &0 &\cdots \\
0 & 0 &  & 0& \ddots & \\
\vdots &  & \ddots &\vdots  &  & -\frac{l_m q_m}{E_m A_m^2} \\
-\frac{l_1 q_1}{E_1 A_1^2} & 0 & \cdots & \frac{l_1 q_1^2}{E_1 A_1^3} \\
0 & \ddots& & & \ddots\\
\vdots & & -\frac{l_m q_m}{E_m A_m^2}&& & \frac{l_m q_m^2}{E_m A_m^3} \\
\end{pmatrix},
\end{align}
which is a symmetric matrix. To see if that the hessian is positive semi-definite we look at its eigenvalues, and calculate $\det(H(g)-\mu I)=0$. It can be shown that this matrix has an LU factorization, since all the main diagonal elements are non-negative, such that we may write $\det(H(g)-\mu I) = \det(L)\det(U) = \text{sign}[\det(L)](-\mu)^{m+3n_s}\Pi_{i=1}^m(\alpha_i-\mu)^2 = 0$, which implies that $\mu \geq 0$, since $\alpha_i > 0$, for all $i$, as defined in Problem 1. Thus, the hessian is positive semi-definite and the objective function is convex, completing the argument that the KKT conditions are sufficient.
\subsection{Barrier functions}
We now restate Problem 1, with the additional constraints in the form of barrier functions. The problem then reads:
\begin{align}\label{barrierproblem}
&\min_{(\boldsymbol{A},\boldsymbol{q},\boldsymbol{f}^\text{supp})} \enspace \frac{1}{2}\sum_{j=1}^m
\frac{l_j q_j^2}{E_j A_j} - \mu \log \left[ M - \sum_{j=1}^m \rho_j l_j A_j \right] \\
\nonumber & \hspace{35 mm} - \mu \sum_{j=1}^m \log (A_j-\underline{A}_j)(\overline{A}_j-A_j),\\
&\text{subject to} \quad B\boldsymbol{q} = I_\text{supp}\boldsymbol{f}^{supp}+I_\text{ext}\boldsymbol{f}^{ext}\nonumber,
\end{align}
where $\mu > 0 $ is a small barrier parameter.
\newpage
\noindent
The non-pertubed KKT conditions for some some solution $\boldsymbol{q}^*,{\boldsymbol{f}^\text{supp}}^*,\boldsymbol{A}^*$ to \eqref{barrierproblem} will be:
\\\\
\textbf{Stationarity}:
\begin{align}
-\frac{1}{2}\text{diag}\left[ \frac{l_j q_j^{*2}}{E_j A_j^{*2}}\right]_{j=1}^m + \left(\frac{\mu}{M-\sum_{j=1}^m \rho_j l_j A_j^*}\right) \text{diag}\left[ \rho_j l_j \right]_{j=1}^m \\
-\mu \ \text{diag}\left[\frac{\underline{A}_j+\overline{A}_j-2A_j^*}{(A_j^*-\underline{A}_j)(\overline{A}_j-A_j^*)}\right]&=0 \nonumber\\
D\boldsymbol{q}^* - \sum_{i=1}^m \lambda_i^*(B)_i^T &= 0,\\
\sum_{i=1}^m \lambda_i^*(I_\text{supp})_i^T &= 0.
\end{align}
\textbf{Primal feasibility}:
\begin{align}
B\boldsymbol{q}^*-I_\text{supp}{\boldsymbol{f}^\text{supp}}^*-I_\text{ext}\boldsymbol{f}^{ext}=0.
\end{align}
Compared to the optimality conditions in \eqref{additionalgay} - \eqref{finalgay}, we see that the convex set satisfying the inequalities have been encoded into the barrier function. In particular, the approximations for the Lagrange multipliers corresponding to inequality constraints in the problem are:
\begin{align}
\lambda_{0\in\mathcal{I}} &\approx \frac{\mu}{M-\sum_{j=1}^m \rho_j l_j A_j^*},\\
\lambda_{j\in\mathcal{I}} &\approx \frac{\mu}{(A_j^*-\underline{A}_j)(\overline{A}_j-A_j^*)}
\end{align}
where $\lambda_0$ is the constraint due to maximal allowable mass, and $\lambda_j$ is the constraint due to the bounds on the area $j$. The KKT conditions are necessary, and if the objective function, with logarithmic barriers, is convex, they would be sufficient by Theorem 2, given some $\mu$. Of course, such solutions, if they exist, would not minimize the original objective function. However, one can show that minimizers of \eqref{barrierproblem} approach a solution of the original objective function as $\mu \rightarrow 0$. 
\\\\
Assuming such a sequence of $\{\mu_k\} \rightarrow 0$ as $k\rightarrow \infty$ exists, which satisfy the necessary KKT conditions for every value of $\mu_k$, and that limits points of the corresponding sequence of feasible iterates $\{ \boldsymbol{f}^\text{supp}_k\},\{\boldsymbol{q}_k\},\{\boldsymbol{A}_k\}$ all satisfy LICQ conditions, then the first-order optimality conditions hold at the limit, by Theorems 19.1, 19.2\footnote{Numerical Optimization 2nd Edition, Page 590}. Given that we have shown the sufficiency of KKT conditions, through convexity of the objective function, without transforming it to a Barrier Problem, this could indicate sufficiency for the Barrier Problem as well. Intuitively, the objective function should lie in the subspace spanned by the constraints gradients, when the Lagrange multipliers are approximated closely.
\section{Conclusion}
We have developed the necessary conditions for which solutions to the truss optimization problem \eqref{eq:min g} may exist. Furthermore, we have proved that these conditions are sufficient to ensure a solution, for both the original problem, and with added constraints. We have also proved that there exists at least one solution to each of these, and that they are unique, given some additional assumptions. Lastly, we have developed the first order necessary conditions for the corresponding barrier problem, compared their Lagrange mulitipliers approximations, and discussed whether or not these conditions may be sufficient.
\begin{thebibliography}{2}
\bibitem{newtonsource}
Numerical Optimization 2nd Edition, Nocedal Jorge, J. Wright Stephen, Springer Series in OPeration Research and Financial Engineering, ISBN-10: 0-387-30303-0
\end{thebibliography}
\end{document}